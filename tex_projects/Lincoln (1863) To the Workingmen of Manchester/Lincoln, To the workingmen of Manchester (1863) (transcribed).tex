\documentclass[12pt]{letter}
\usepackage{hyperref}

\usepackage{geometry}
\geometry{a5paper}
\usepackage{lmodern}

\address{\vspace{-2em}}
\date{Executive Mansion, \\ 
	Washington, January, 19, 1863.}
\begin{document}

\begin{letter}{}
\opening{To the Workingmen of Manchester.}

 \marginpar{\textbf{[1]}} I have the honor to acknowledge the receipt of the address and resolutions which you sent to me on the eve of the New Year.

When I came, on the fourth day of the March, 1861, through a free and constitutional election, to preside in the Government of the United States, the country was found at the verge of civil war. Whatever might have been the cause, or whosoever the fault, one duty, paramount to all others, was before me, namely: to maintain and preserve at once the constitution and the integrity of the Federal Republic. A conscientious purpose to perform this duty, is the key to all the measures of administration \marginpar{\textbf{[2]}} which have been, and to all which will hereafter be pursued. Under our frame of Government, and my official oath, I could not depart from this purpose if I would. It is not always in the power of Government to enlarge or restrict the scope of moral results, which follow the policies that they may deem it necessary, for the public safety, from time to time, to adopt.

I have understood well that the duty of self-preservation rests solely with the American People. But I have, at the same time, been aware that favor or disfavor of foreign nations might have a material influence in enlarging and prolonging the struggle which disloyal men in which the country is engaged. A fair examination of history has seemed to authorize a belief that the past action and influences of the United States were generally regarded as having been beneficent towards mankind. I have therefore reckoned upon the forbearance of nations. Circumstances, to some of which you \marginpar{\textbf{[3]}} kindly allude induced me especially to expect that, if justice and good faith should be practised by the Unite dStates, they would encounter no hostile influence on the part of Great Britain. It is now a pleasant duty to acknowledge the demonstration you have given of your desire that a spirit of peace and amity towards this country may prevail in the councils of your Queen, who is respected and esteemed in your own country only more than she is by the kindred nation which has its home on this side of the Atlantic.

I know and deeply deplore, the suffering which the workingmen at Manchester, and in all Europe, are called to ensure in this crisis. It has been often and studiously represented that the attempt to overthrow this Government, which was built upon the foundation of Human Rights, and to substitute for it one which should rest exclusively on the basis of Human Slavery, was likely to obtain the favor of Europe. Through the action of our disloyal citizens, the \marginpar{\textbf{[4]}} workingmen of Europe have been subjected to a severe trial, for the purpose of forcing their sanction to that attempt. Under the circumstances, I cannot but regard your decisive utterances upon the question as an instance of sublime Christian heroism, which has not been surpassed in any age or in any country. it is indeed an energetic and inspiring assurance of the inherent power of truth, and of the ultimate and universal triumph of Justice, Humanity and Freedom. I do not doubt that the sentiments you have expressed will be sustained by your great nation; and on the other hand, I have no hesitation in assuring you that they will excite admiration, esteem, and the most reciprocal feelings of friendship among the American People. I hail this interchange of sentiment, therefore, as an augury that whatever else may happen, whatever misfortune may befall your country or my own, that peace and friendship which now exists between the two nations will be, as it shall be my desire to make them, perpetual.

\signature{\vspace{-4em}Abraham Lincoln}
\closing{(sd.)}


\end{letter}
\end{document}