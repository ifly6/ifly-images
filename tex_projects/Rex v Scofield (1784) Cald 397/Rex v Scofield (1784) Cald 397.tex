\documentclass[a4paper]{article}
\usepackage[utf8]{inputenc}
\usepackage[british]{babel}

\usepackage[T1]{fontenc}
\usepackage{lmodern}

% \usepackage{charter}

\usepackage[hang,norule,multiple]{footmisc}
\usepackage[style=british]{csquotes}
\usepackage[allbordercolors={0 0.8 0}]{hyperref}

\newcommand{\etc}{\emph{\&}c\ }

\title{\vspace{-6ex} Rex v Scofield \\
{\large (1784) Cald 397}}
\author{\vspace{-6ex}}
\date{\vspace{-6ex}}

\begin{document}

\maketitle

Free version of the original can be  \href{https://play.google.com/books/reader?id=ZFcDAAAAQAAJ&hl=en_GB&pg=GBS.PA396}{found on Google Play}. The text is modernised so that a person in the modern day can read it. I did not however, modernise the sentence structure that Caldecott uses, which is why he seemingly only knows the word `said' and puts it in every sentence. Errors are mine.

The citation form is \ldots\ aged. Cases are first given by term: eg Trinity, Easter, Michaelmas, Hilary. Then the regnal year. For monarchs that have the same name, you have abbreviations like `G 3' standing in for `George III'. Following this is the reporter. All reporters are nominative, ie associated with a person by name. At the end is either a page or a folio.

The reference to `H P C 2.~172' probably refers to \href{https://en.wikipedia.org/wiki/Historia_Placitorum_Coron\%C3\%A6}{\emph{Historia Placitorum Coronae}}, written by Sir Matthew Hale, volume 2 at page 172. I cannot confirm this, however, as I do not have a copy of this pre-Blackstone legal commentary.

Page numbers of the original are in bold in margin paragraphs. I have attempted to preserve original formatting.

$$- * -$$

\marginpar{\textbf{397}} \marginpar{\emph{Wednesday Feb 4}} This was an indictment for arson,\footnote{Arson is an injury only to the actual possession and must be so laid. Where upon an indictment an act is charged to have been committed feloniously, and the jury find a verdict of guilty, through the charge, though the charge, as laid, does not amount of felony, yet if it does amount in law to a misdemeanour, the Court will pronounce judgement as for that offence.} tried before Lord \emph{Mansfield}, at the sittings for \emph{Westminster} after last \emph{Michaelmas} term. It contained six counts.

The first count stated that \emph{John Scofield} of the parish of \emph{St George Hanover Square} in the county of \emph{Middlesex}, labourer, wickedly unlawfully and maliciously intending devising and contriving to \marginpar{\textbf{398}} feloniously set fire to, burn and consume a certain house of one \emph{James Ramsay} there situated (of which house he the said \emph{John Scofield was then possessed for a certain term of years} then and yet to come and unexpired) on, \etc with force and arms at \etc a certain lighted wax candle, which he the said \emph{John Scofield} had then lately before set fire to and lighted did unlawfully wickedly and maliciously fix and put in a certain closet under and adjoining certain wooden stairs called the kitchen stairs in the aforesaid house of the said \emph{James Ramsey}, which said house was then and now is situate and being in a certain neighbourhood and street there called \emph{New Bond Street} and \emph{contiguous and adjoining to certain dwelling houses} thereof, and belonging to divers and of the liege subjects of our said lord the King : And that he the said \emph{John Scofield} did then and there unlawfully wickedly and maliciously put and place about unto and against the said lighted candle so fixed and put by him the said \emph{John Scofield} in the said closet as aforesaid divers matches and small pieces of wood and other combustible materials \emph{with a wicked and malicious intention by means thereof then and there feloniously to set fire to the aforesaid house of the said James Ramsay} and to burn and consume the same to the great damage, \etc.

The second count no otherwise varied the charge than by describing the house to be the \emph{dwelling} house of the said \emph{John Scofield}.

The third count stated that the prisoner set fire to certain matches and small pieces of wood \etc in a certain other house of the said \emph{James Ramsay} \etc under certain wooden stairs \etc by means thereof feloniously to set fire to the said \emph{last mentioned house} \etc : without stating, that it was in the possession of the prisoner, or that he had any term in it.

The fourth count charged the offence to have been committed in the same manner, in house of the \emph{prisoner}.

The fifth count charged an attempt to set fire to the house of the said \emph{James Ramsay}, and

The sixth count an attempt to set fire to the house of the \emph{prisoner}.

All counts laid the offence to have been done \emph{feloniously} ; but none of them charged an intent of setting fore to the adjoining houses.

The jury found the prisoner guilty.

Several objections had been taken at the trial, but Lord \emph{Mansfield} declined giving any opinion upon them, declaring at the time to the prisoner's counsel, that they should have the full benefit of them before the court ; though some of them were not altogether objections \marginpar{\textbf{399}} in point of law, but went to show that the evidence did not support some of the charges.

\emph{Silvester} and \emph{Mingay} made two objections in arrest of judgement. That this indictment, having charged the offence to have been done feloniously, could only be supported by showing it to be a felony : and 2. That if it was not a felony, there was no offence charged ; an attempt to commit a misdemeanour not falling under any class or denomination of indictable subjects : and they insisted that to set fire to one's own house is not a felony : that this had in a very late case, that of \emph{K v Pedley},\footnote{Tr 22 G 3. 1782 ante 218.} been fully settled.

\hspace{\parindent} Lord \emph{Mansfield}.

But on wretched reasoning.

\emph{Silvester} and \emph{Mingay}. That to constitute a crime here, an intent to burn the house of another must be alleged ; but that no such charge had been made ; no neighbours are injured, no insurances had been affected ; no charge of this description is alleged ; that no felony being stated, the judgement ought to be arrested : that the Court could not reject the word feloniously ; for that the prosecutor is bound to state his charge with precision, left the Court be led to pronounce sentence for a more aggravated offence than the delinquent is guilty of : that one felony may be charged, and the prisoner may be convicted of a felony of another denomination, being another modification, of a less criminal view, of the same act ; as in murder, burglary, breaking and entering the house, and larceny : but that is no case, containing the charge of a \emph{crime} ie anything felonious, could there be a conviction for a misdemeanour ; or where one misdemeanour is charged a conviction for another : that the only charge which this indictment contained was a contradiction to itself ; for that, if this act was not a felony, it was any offence : that to do an innocent act with intent to commit a misdemeanour is not an indictable offence : that here this would but have been a misdemeanour, if the act had been completed : and that an attempt to commit a misdemeanour is not a misdemeanour.

\hspace{\parindent} Lord \emph{Mansfield}.

Do you find any authority, that warrants this proposition?

\emph{Bearcroft} and \emph{Rose} insisted in support of the prosecution ; that it must be a misdemeanour to set fire to one's own house, when it adjoins other houses ; because such an act endangers the public safety : that it had been established that an attempt to commit a \marginpar{\textbf{400}} misdemeanour was punishable as a misdemeanour : that in an indictment at \emph{Shrewsbury}, which charged the defendant before \emph{Adams B} with an attempt to stubborn a man to commit perjury, it had been holden by all the judges, to whom the case was referred, that it was a misdemeanour ; and the defendant received a heavy sentence : and that in the case of \emph{Bush v Rawlins},\footnote{Mentioned in 3 Burr fo[lio] 1236.} the giving a bribe to forbear voting at an election for an adversary, though the party bribed did not forbear, was holden an offence.

\hspace{\parindent} \emph{Buller} J.

That was made an offence by the Act of Parliament.\footnote{2 Geo 2. c 24 \S\ 7 (modern title Corrupt Practices at Parliamentary Elections Act 1728).}

\emph{Bearcroft}. There is also the case of the \emph{K v Vaughan},\footnote{M 10 Geo 3. 1769. 4 Burr 2494.} which was an attempt to bribe a minister to recommend to an office : and of the \emph{K v Johnson},\footnote{30 Car 2. 2 Show 1.} which was an attempt by an attorney to bribe a witness.

\hspace{\parindent} Lord \emph{Mansfield} and \emph{Buller} J.

It makes a great difference, whether an act was done ; as in this case putting fire to a candle in the midst of combustible matter, (which was the only act necessary to commit a misdemeanour) and where no act at all is done. the \emph{intent} may make an act, innocent itself, criminal ; nor is the \emph{completion} of an act, criminal in itself, necessary to constitute criminality. Is it no offence to set fire to a train of gunpowder with intent to burn a house, because by accident, or the interposition of another, the mischief is prevented?

\hspace{0.7\textwidth} \emph{Cur advisare vult.}

\marginpar{\emph{Wednesday, Feb 11}} And now Lord \emph{Mansfield} delivered the judgement of the court.

After stating the indictment his Lordship observed, that the third count was clear of all possible objection ; for there, without alleging that the house is in the prisoner possession, it is said to be the house of \emph{James Ramsay} ; and, as it would have been a felony in the prisoner to have burnt such a house, the intent was there properly charged to \emph{felonious}. On that count therefore judgement might have been given ; but as on that the evidence did not support the charge, we shall found our judgement on the first count.

Two objections have been made. 1. That this act is laid to have been done \emph{feloniously}, and, as that word cannot be rejected, and not felony is charged, the defendant being the occupier of the house, the intent could not have been felonious ; and the indictment \marginpar{\textbf{401}} consequently is throughout bad : 2. that if this word is rejected, the indictment will still be bad, as it will then charge no offence at all : it merely being an attempt to commit a misdemeanour.

The defendant's counsel begin by insisting upon what is not settled : that it could be no felony in the defendant to \emph{burn} a house, of which he was in possession ; and therefore that it could not be felony only to \emph{intend} to do so : and so far is true. But they urge further, that the word `feloniously' cannot be rejected, and also that the defendant cannot be convicted of less than a felony, ie of a misdemeanour, which it appears on the face of the indictment that felony is charged; and so say they, there is no offence, and there can be no judgement.

This argument is a contradiction in itself ; for it at the same time says, that a felonious intent is and is not charged. Where a particular intent is charged, the whole must be taken together and the law must so judge of it. Here the intent charged is, that he feloniously attempted to burn his own house ; but the law says, that is not a felony ; and therefore, this being an averment repugnant to the legal import of the offence charged, the word `feloniously' must be rejected. To this the \emph{King v Holmes},\footnote{M 10 Car Cr Car 376. Sir W Jones 351.} is an authority expressly in point : and, when examined and rightly understood, is not liable to the objection made to it be Lord \emph{Hale},\footnote{H P C 2. 172.} who states it thus---
\begin{quote}
    A man indicted for felonious burning of a house, upon \emph{not guilty} pleaded, a special verdict was found, it was adjudged no felony, as the case was found, yet upon the same indictment he was adjudged to the pillory, and fined 500 \emph{l} and bound to his good behaviour, but \emph{quaere} of that case, for it seems unreasonable, because being tried for felony, he hath not those advantages for his defence, as if he were indicted only for trespass.
\end{quote}

He supposes the case properly charged and that there was a special verdict, and that there the Court gave judgement as for another crime. Had the case been so, it might have well deserved the \emph{quaere} put by Lord \emph{Hale} : But it is quite mistaken ; for there was no special verdict. The jury found a general verdict of guilty. The indictment was removed here by \emph{certiorari}, and there was nothing before the Court, but the indictment. The indictment charged, that \emph{Holmes}, being possessed of a house in \emph{London}, did feloniously set on fire his own house and burn it with the intent to burn the houses \marginpar{\textbf{402}} of others adjoining. The whole argument was confined to the question of what was charged in the imdictment. The Court were of opinion, that notwithstanding it was charged to have been done `feloniously', that it was only a misdemeanour, and they gave judgement as for a misdemeanour ; the question being whether a felony or not? This is apparent in the report of the case by \emph{Crooke} J and in \emph{Joyner}'s case\footnote{16 Car 2. Kel 29.} in 1664. \emph{Kelying} Ch J and \emph{Wylde} J in differing from \emph{Hyde} Ch J state, that in \emph{Holmes} case `all the special matter was expressed in the indictment'. It is therefore a conclusive authority : and on principle and reason it appears to us to be mostly clearly right. This being so,

The next question is, Whether an act done in pursuance of an intent to commit an act, which, if completed, would be a misdemeanour only, can itself be a misdemeanour? It was objected, that an attempt to commit a misdemeanour was no offence : but no authority for this is cited ; and there are many on the other side : as the case cited ; of the \emph{King v Johnson}, the \emph{King v Sutton},\footnote{E 10 G 2. 2 Str 1074.} which was an indictment for having in custody and possession stamps \emph{with intent} to impress sceptres on sixpences \etc. And there the court say lading wool is lawful ; but if it be with \emph{an intent} to transport it, that makes it an offence. Here the intent is the offence ; and the having in his custody, an act that is the evidence of that intent'. But in the case of the wool, the transporting of it was only a misdemeanour, yet an act done to that end was holden indictable. In the \emph{King v Taylor},\footnote{Tr 15 G 2. 2 Str 1167.} the Court granted an information as for a nuisance for keeping great quantities of gunpowder to the endangering o the church and houses where the defendant lived. There is also the case cited of the \emph{King v Samuel Vaughan}, which is founded on the same principle as that of the \emph{King v Plympton}\footnote{M 11 G 2. 1724. 2 Ld Raym 1377.} ; where it was holden that to bribe a corporator by money or promises to vote at corporation elections is an offence, for which an information will lie : the case of \emph{Vaughan} was that of offering a bribe for an office, and if received, and the office procured, neither parry would have been guilty of more than a misdemeanour : and it is laid down by the Court in the case of the \emph{King v Langley}\footnote{H 2 Ann. 2 Salk 697.} that words directly tending to a breach of the peace are indictable.

\marginpar{\textbf{403}} There was a distinction made at the bar between an act done with an intent to commit a felony and an act done with an intent to commit a misdemeanour. In the degrees of guilt there is great difference in the eye of the law, but not in the description of the offence. So long as an act rests in bare intention, it is not punishable by our laws : but immediately when an act is done, the law judges, not only of the act done, but of the intent with which it is done ; and, if it is coupled with an unlawful and malicious intent, though the act itself would otherwise have been innocent, the intent being criminal, the act becomes criminal and punishable. The case cited of the \emph{King v Sutton} is an express authority. We are therefore of opinion that the indictment is good.

\hspace{0.7\textwidth} Rule discharged.

Lord \emph{Mansfield} then reported the evidence : and \emph{Willes} J pronounced the judgement of the court, which was, that the prisoner pay a fine of 300 \emph{l}, be imprisoned in \emph{Newgate} one year till his fine be paid ; and that he give security for his good behaviour for 7 years, himself in 200 \emph{l} and two sureties in 100 \emph{l} each.

\end{document}
