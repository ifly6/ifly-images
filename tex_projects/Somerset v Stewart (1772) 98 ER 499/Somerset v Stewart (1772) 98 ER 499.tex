\documentclass[a4paper]{article}
\usepackage[utf8]{inputenc}
\usepackage[british]{babel}

\usepackage[T1]{fontenc}
\usepackage{lmodern}

% \usepackage{charter}

\usepackage[hang,norule,multiple]{footmisc}
\usepackage[style=british]{csquotes}
\usepackage[allbordercolors={0 0.8 0}]{hyperref}

\newcommand{\etc}{\emph{\&}c\ }

\title{\vspace{-6ex} \emph{Someset v Stewart} \\
{\large (1772) 98 ER 499}}
\author{\vspace{-6ex}}
\date{\vspace{-6ex}}

\begin{document}

\maketitle

Free version of the original can be  \href{http://www.commonlii.org/int/cases/EngR/1772/57.pdf}{found on Common LII}.

The citation form is \ldots\ aged. Cases are first given by term: eg Trinity, Easter, Michaelmas, Hilary. Then the regnal year. For monarchs that have the same name, you have abbreviations like `G 3' standing in for `George III'. Following this is the reporter. All reporters are nominative, ie associated with a person by name. At the end is either a page or a folio.

Page numbers of the original are in bold in margin paragraphs. I have attempted to preserve original formatting. All section headings are mine and not present in the original.

The view of Mansfield at the end is that `the black must be discharged', meaning that Somerset must be discharged from his confinement aboard Captain Knowles' ship. Lord Mansfield later interpreted his decision as meaning that a slave could not be removed from England against the slave's will; he did not interpret it to mean that all slaves were freed.\footnote{\emph{R v Thames Ditton}, (1795) 99 ER 891.}

The slave trade would become illegal in 1807 with the passage of the Slave Trade Act. The Slave Trade Felony Act 1811 made it a felony to trade in slaves. Slavery Abolition Act 1833 abolished slavery over a period of seven years with compensation to be paid to the slave owners.

\tableofcontents

\clearpage

\section{Headnotes}
\begin{center}
    \textsc{Easter Term, 12 Geo. 3, 1772, K. B.} \marginpar{\textbf{499}}
    
    \textsc{Somerset} \emph{against} \textsc{Stewart}. May 14, 1772.
\end{center}

On return to an habeas corpus, requiring Captain Knowles to show cause for the seizure and detainure of the complainant Somerset, a negro---the case appeared to be this---

That the negro had been a slave to Mr Stewart, in Virginia, had been purchased from the African coast, in the course of the slave-trade, as tolerated in the plantations ; that he had been brought over to England by his master, who intending to return, by force sent him on board of Captain Knowles's vessel, lying in the river ; and was there, by the order of his master, in the custody of Captain Knowles, detained against his consent ; until returned in obedience to the writ. And under this order, and the facts stated, Captain Knowles relied in his justification.

Upon the second argument, (Serjeant Glynn was in the first, and I think, Mr Mansfield) the pleading on behalf of the negro was opened by Mr Hargrave. I need not say that it will be found at large, and I presume has been read by most of the profession, he having obliged the public with it himself : but I hope this summary note, which I took of it at the time, will not be though impertinent ; as it is not easy for a cause in which that gentleman has appeared, not to be materially injured by a total omission of his share in it.

\section{Plaintiff I}

\subsection{Mr Hargrave}
Mr Hargrave.---The importance of the question will I hope justify to your Lordships the solicitude with which I rise of defend it ; and however unequal I feel myself, will command attention. I trust, indeed, that this is a cause sufficient to support my own \textbf{[2]} unworthiness by its single intrinsic merit. I shall endeavour to state the grounds from which Mr Stewart's supposed right arises ; and then offer, as appears to me, sufficient confutation to his claim over the negro, as property, after having him brought over to England ; (an absolute and unlimited property, or as right accruing from contract ; ) Mr Stewart insists on the former. The question on that is not whether slavery is lawful in the colonies, (where a concurrence of unhappy circumstances has caused it to be established as necessary ; ) but whether in England? Not whether it \marginpar{\textbf{500}} ever has existed in England ; but whether it be not now abolished? Various definitions have been given of slavery : one of the most considerable is the following ; a service for life, for bare necessaries. Harsh and terrible to human nature as even such a condition is, slavery is very insufficiently defined by these circumstances---it includes not the power of the master over the slave's person, property, and limbs, life only excepted ; it includes not the right over all acquirements of the slave's labour ; nor includes the alienation of the unhappy object from his original master, to whatever absolute lord, interest, caprice or malice, may choose to transfer him ; it includes not the descendible property from father to son, and in like manner continually of the slave and all his descendants. Let us reflect on the consequences of servitude in a light still more important. The corruption of manner in the master, from the entire subjection of the slaves he possesses to his sole will ; from whence spring forth luxury, pride, cruelty, with the infinite enormities appertaining to their train ; the danger to the master, from the revenge of his much injured and unredressed dependant ; debasement of the mind of the slave, for want of means and motives of improvement ; and peril to the constitution under which the slave cannot but suffer, and which he will naturally endeavour to subvert, as the only means of retrieving comfort and security to himself.---The humanity of modern times has much mitigated this extreme rigour of slavery ; shall an attempt to introduce perpetual servitude here to this island hope for countenance? Will not all the other mischiefs of mere utter servitude revive, if once the idea of absolute property, under the immediate sanction of the laws of his country, extend itself to those who have been brought over to a soil whose air is deemed too pure for slaves to breathe in it ; but the laws, the genius and spirit of the constitution, forbid the approach of slavery ; will not suffer its existence here. This point, I conceive, needs no further enlargement : I mean, the proof of our mild and just constitution is ill adapted to the reception of arbitrary maxims and practices. But it has been said by great authorities, though slavery in its full extent be incompatible with the natural rights of mankind, and the principles of good government, yet a moderate servitude may be tolerated ; nay, sometimes must be maintained. Captivity in war is the principal ground of slavery : contract another. Grotius De \textbf{[3]} J. B. \& P. and Pufendorf, b. 6, c. 3, \S\ 5, approves of making slaves of captives in war.\footnote{Ed. he refers to Hugo Grotius, `De jure belli ac pacis' (1625) book 6 ch 3 s 5. The Latin translates to `On the law of war and peace'.} The author of the Spirit of Laws denies, except for self-preservation, and then only a temporary slavery. Dr Rutherforth, in his Principles of Natural Law, and Locke, absolutely against it. As to contract ; want of sufficient consideration justly gives full exception to the considering of it as contract. If it cannot be supported against parents, certainly not against children. Slavery imposed for the performance of public works for civil crimes, is much more defensible, and rests on quite different foundations. Domestic slavery, the object of the present consideration, is now submitted to observation in the ensuing account, its first commencement, progress, and graduate decrease : it took origin very early among the barbarous nations, continued in the state of the Jews, Greeks, Romans, and Germans ; was propagated by the last over the numerous and extensive countries they subdued. Incompatible with the mild and human precepts of Christianity, it began to be abolished in Spain, as the inhabitants grew enlightened and civilised, in the 8th century ; its decay extended over Europe in the 4th ; was pretty well perfected in the beginning of the 16th century. Soon after that period, the discovery of America revived those tyrannic doctrines of servitude, with their wretched consequences. There is now at last an attempt, and the first yet known, to introduce it into England ; long and uninterrupted usage from the origin of the common law, stands to oppose its revival. All kinds of domestic slavery were prohibited, except villenage. The villain was bound indeed to perpetual service ; liable to the arbitrary disposal of his lord. There were two sorts ; villain regardant, and in gross : the former as belonging to a manor, to the lord of which his ancestors had done villain service ; in gross, when a villain was granted over by the lord. Villains were originally captives at the Conquest, or troubles before. Villenage could commence no where but in England, it was necessary to have prescription for it. A new species has never arisen till now ; for had it, remedies and powers there would have been at law : therefore the most violent presumption against is the silence of the laws, where there nothing more. 'Tis very doubtful whether the laws of England will permit a man to bind himself by contract to serve for life : certain will not suffer him to invest another man with despotism, nor prevent his own right to dispose of property. If disallowed by consent of parties, much more when by force ; if made void wen commenced here, \marginpar{\textbf{501}} much more when imported. If these are true arguments, they reach the King himself as well as the subject. Dr Rutherforth says, if the civil law of any nation does not allow of slavery, prisoners of war cannot be made slaves. If the policy of our laws admits not of slavery, neither fact nor reason are for it. A man, it is said, told the Judges of the Star Chamber, in the case of a Russian slave whom they had ordered to be scourged and imprisoned, that the air of England was too pure for slavery. The Parliament afterwards punished the Judges of the Star Chamber for such usage of the \textbf{[4]} Russian, on his refusing to answer interrogatories. There are very few instances, few indeed, of decisions as to slaves, in this country. Two in Charles the 2d, where it was adjudged trover would lie.\footnote{Ed. Trover meaning a common law action to recover the value of personal property wrongfully disposed of by another. It was abolished with the end of the forms of action.} \emph{Chamberlayne and Perrin}, Will. 3d, trover brought for taking a negro slave, adjudged it would not lie.---4th Ann. action of trover ; judgement by default : on arrest of judgement, resolved that trover would not lie. Such the determinations in all but two cases ; and those the earliest, and disallowed by the subsequent decisions. Lord Holt.---As soon as a slave enters England he becomes free. \emph{Stanley and Harvey}, on a bequest to a slave ; by a person whom he had served some years by his former master's permission, the master claims the bequest ; Lord Northington decides for the slave, and gives him costs. 29th of George the 2d, c. 31,\footnote{Ed. Wikipedia says that this is Infants, Lunatics, etc. Act 1756. I am not sure about that.} implies permission in America, unhappily thought necessary ; but the same reason subsists not here in England. The local law to be admitted when no very great inconvenience would follow ; but otherwise not. The right of the master depends on the condition of slavery (such as it is) in America. If the slave be brought hither, it has nothing left to depend on but a supposed contract of the slave to return ; which yet the law of England cannot permit. Tush has been traced the only mode of slavery ever established here, villenage, long expired ; I hope it has shewn, the introducing new kinds of slavery has been cautiously, and we trust, effectually guarded against by the same laws. Your Lordships will indulge me in reciting the practice of foreign nations. 'Tis discounted in France ; Bartholinus De Republica denies its permission by the law of France. Molinus gives a remarkable instance of the slave of an ambassador of Spain brought into France : he claims liberty ; his claim allowed. France even mitigates the ancient slavery, far from creating new. France does not suffer even her King to introduce a new species of slavery. The other Parliaments did indeed ; but the Parliament of Paris, considering the edict to import slavery as an exertion of the Sovereign to the breach of the constitution would not register that edict.\footnote{Ed. The French \emph{parlements} were regional courts that had legislative functions in that royal edicts would not take effect until the \emph{parlement} registered it within their jurisdiction. They were not fully empowered legislative bodies like the British Parliament.} Edict 1685, permits slavery in the colonies. Edict in 1716, recites the necessity to permit in France, but under various restraints, accurately enumerated in the Institute of French Laws. 1759 Admiralty Court of France ; causes Celebr\`ees, title Negro. A French gentleman purchased a slave, and sent him to St Malo's entrusted with a friend. He came afterwards and took him to Paris. After ten years the servant chooses to leave France. The master not like Mr Stewart hurries him back by main force, but obtains a process to apprehend him, from a Court of Justice. While in prison, the servant institutes a process against his master, and is declared free. After the permission of slaves in the colonies, the dict of 1716 was necessary, to transfer that slavery to Paris ; not without many restraints, as before remarked ; otherwise the ancient principles would have prevailed. The author De Jure Novissimo, though the natural tendency of his book, as appears by the title, leads the other way, concurs with \textbf{[5]} diverse great authorities, in reprobating the introduction of a new species of servitude. In England, where freedom is the grand object of the laws, and dispensed to the meanest individual, shall the laws of an infant colony, Virginia, or of a barbarous nation, Africa,\footnote{Ed. Here the barrister is likely referring to the Barbary States in northern Africa, which at the time were known from capturing Europeans from ships they raided and pressing them into servitude.} prevail? From the submission of the negro to the laws of England, he is liable to all their penalties, and consequently has a right to their protection. There is one case I must still mention ; some criminals having escaped execution in Spain, were set free in France. [Lord Mansfield.---Rightly : for the laws of one country has not whereby to condemn offences supposed to be committed against those of another.]

An objection has arisen, that the West India Company, with their trade in slaves, having been established by the law of England, its consequences must be recognised by the law ; but the establishment is local, and these consequences local ; and not the law of England, but the law of the plantations.

The law of Scotland annuls the contract to serve for life ; except in the case of colliers, and one other instance of a similar nature.\footnote{Ed. A collier is a coal miner.} A case is to be found in the History of the Decision, where a term of years was discharged, as exceeding the usual limits of human life. As least, if contrary to all these decisions, the Court \marginpar{\textbf{502}} should incline to think Mr Stewart has a title, it must be presumption of contract, there being no deed in evidence : on this supposition, Mr Stewart was obliged, undoubtedly, to apply to a Court of Justice. Was it not sufficient, that without form, without written testimony, without even probability of a parole contract, he should venture to pretend to a right over the person and property of the negro, emancipated, as we contend, by his arrival hither, at a vast distance from his native country, while he vainly indulged the natural expectation of enjoying liberty, where there was no man who did not enjoy it? Was not this sufficient, but he must still proceed, seize the unoffending victim, with no other legal pretence for such a mode of arrest, but the taking an ill advantage of some inaccurate expressions in the Habeas Corpus Act\footnote{Habeas Corpus Act 1679; 31 Cha 2 c 2.} ; and thus pervert an establishment designed for the perfecting of freedom? I trust, an exception from a single clause, inadvertently worded, (as I must take the liberty to remark again) of that one statute, will not be allowed to overrule the law of England. I cannot leave the Court, without some excuse for the confusion in which I rose, and in which I now appear : for the anxiety and apprehension I have expressed, and deeply felt. It did not arise from want of consideration, for I have considered this case for months, I may say years ; but less did it spring from a doubt, how the cause might recommend itself to the candour and wisdom of the Court. But I felt myself overpowered by the weight of the question. I now, in full \textbf{[6]} conviction how opposite to natural justice Mr Stewart's claim is, in firm persuasion of its inconsistency with the laws of England, submit it cheerfully to the judgement of his honourable Court : and hope as much honour to your Lordships from the exclusion of this new slavery, as our ancestors obtained from the abolition of the old.

\subsection{Mr Alleyne}
Mr Alleyne.---Though it may seem presumption in me to offer any remarks, after the elaborate discourse but now delivered, yet I hope the indulgence of the Court ; and shall confine my observations to some few points, not included by Mr Hargrave. 'Tis well known to your Lordships, that much has been asserted by the ancient philosophers and civilians, in defence of the principles of slavery : Aristotle has particularly enlarged on that subject. An observation still it is, of one of the most able, most ingenious, most convincing writers of modern times, whom I need not hesitate, on this occasion, to prefer to the great Aristotle, the great Montesquieu, that Aristotle, on this subject, reasoned very unlike the philosopher. He draws his precedents from barbarous ages and nations, and then deduces maxims from them, for the contemplation and practice of civilised times and countries. If a man who in battle has had his enemy's throat at his sword point, spares him, and says therefore he has power over his life and liberty, is this true? By whatever duty he was bound to spare him in battle, (which he always is, when he can with safety) by the same he obliges himself to spare the life of the captive, and restore his liberty as soon as possible, consistent with those considerations from whence has was authorised to spare him at first ; the same indispensable duty operates throughout. As a contract : in all contracts there must be power on one side to give, on the other to receive ; and a competent consideration. Now, what power can there be in any man to dispose of all the rights vested by nature and society in him and his descendants? He cannot consent to part with him, without ceasing to be a man ; for they immediately flow from, and are essential to, his condition as such : they cannot be taken from him, for they are not his, as a citizen or a member of society merely ; and are not to be resigned to a power inferior to that which gave them. With respect to consideration, what shall be adequate? As a speculative point, slavery may a little differ in its appearance, and the relation of master and slave, with the obligations on the part of the slave, may be conceived ; and merely in this view, might be thought to take effect in all places alike ; as natural relations always do. But slavery is not a natural, 'tis a municipal relation ; an institution therefore confined to certain places, and necessary dropped by passage into a country where such municipal regulations do not subsist. The negro making choice of his habitation here, has subjected himself to the penalties, and is therefore entitled to the protection of our laws. One remarkable case seems to require being mentioned : some Spanish criminals having escaped from execution, where set free in France. [Lord Mansfield.---Note the distinction in the case : in this case, \textbf{[7]} France was not bound to judge by the municipal laws of Spain ; nor was to take cognisance of the offences supposed against that law.] There has been started an objection, that a company having been established by our Government for the trade of slaves, it were unjust to deprive them here.---No : the Government incorporated \marginpar{\textbf{503}} them with such powers as individuals has used by custom, the only title on which that trade subsisted ; I conceive, that had never extended, nor could extend, to slaves brought hither : it was not enlarged at all by the incorporation of that company, as to the nature or limits of its authority. 'Tis said, let slaves know they are all free as soon as they arrive here, they will flock over in vast numbers, overrun this country, and desolate the plantations. There are too strong penalties by which they will be kept in ; nor are the persons who convey them over much induced to attempt it ; the despicable condition in which negroes have the misfortune to be considered, effectually prevents their important in any considerable degree. Ought we not, on our part, to guard and preserve that liberty by which we are distinguished by all the earth! to be jealous of whatever measure has a tendency to diminish the veneration due to the first of blessings? The horrid cruelties, scare credible in recital, perpetrated in America, might, by the allowance of slaves amongst us, be introduced here. Could your Lordship, could any liberal and ingenuous temper indure, in the fields bordering on this city, to see a wretch bound for some trivial offence to a tree, torn and agonising beneath the scourge? Such objects might by time become familiar, become unheeded by this nation ; exercised, as they are now, to far different sentiments, may those sentiments never be extinct! the feelings of humanity! the generous sallies of free minds! May such principles never be corrupted by the mixture of slavish customs! Nor can I believe, we shall suffer any individual living here to want that liberty, whose effects are glory and happiness to the public and every individual.

\section{Defendant}
\subsection{Mr Wallace}
Mr Wallace.---The question has been stated, whether the right can be supported here ; or if it can, whether a course of proceedings at law be not necessary to give effect to the right? 'Tis found in three quarters of the globe, and in part of the fourth. In Asia the whole people ; in Africa and America far the greater part ; in Europe great numbers of Russians and Polanders. As to captivity in war, the Christian princes have been used to give life to the prisoners ; and it took rise probably in the Crusades, when they have them life, and sometimes franchised them, to enlist under the standard of the Cross against the Mahometans.\footnote{Ed. Archaic spelling of Mohammad.} The right of a conqueror was absolute in Europe, and it is in Africa. The natives are brought from Africa to the West Indies ; purchase is made there, not because of positive law, but there being no law against it. It cannot be in consideration by this or any other Court, to see, whether the \textbf{[8]} West India regulations are the best possible ; such as they are, while they continue in force as laws, they must be adhered to. As to England, not permitting slavery, there is no law against it ; nor do I find any attempt has been made to prove the existence of one. Villenge itself has all but the name. Though the dissolution of monasteries, amongst the other material alterations, did occasion the decay of that tenure, slaves could breathe in England : for villains were in this country, and were mere slaves, in Elizabeth. Sheppard's Abridgment, afterwards, says they were worn out in his time. [Lord Mansfield mentions as assertion, but does not recollect the author, that two only were in England in the time of Charles the 2d, at the time of the abolition of tenures.] In the cases cited, the two first directly affirm an action of trover, an action appropriated to mere common chattels. Lord Holt's opinion, is a mere dictum, a decision unsupported by precedent. And if it be objected, that proper action could not be brought, 'tis a known and allowed practice in mercantile transactions, if the cause arises abroad, to law it within the kingdom : therefore the contract in Virginia might be laid to be in London, and would not be traversable. With respect to the other cases, the particular mode of action was alone objected to ; had it been an action per quo servitium amisit, for the loss of service, the Court would have allowed it. The Court called the person, for the recovery of whom it was brought, a slavish servant, in \emph{Chamberlayne's case}. Lord Hardwicke, and the afterwards Lord Chief Justice Talbot, then Attorney and Solicitor General, pronounced a slave not free by coming into England. 'Tis necessary the masters should bring them over ; for they cannot trust the whites, either with the stores or the navitagin the vessel. Therefore, the benefit taken on the Habeas Corpus Act ought to be allowed.

Lord Mansfield observes, the case alluded to was upon a petition in Lincoln's Inn Hall after dinner ; probably, therefore, might not, as he believes the contrary is not usual at that hour, be taken with much accuracy. The principal matter was then ,on the earnest solicitation of many merchants, to know, whether a slave was freed by being made a Christian? And it was resolved, not. 'Tis remarkable, tho' the English took \marginpar{\textbf{504}} infinite pains before to prevent their slaves being made Christians, that they might not be freed, the French suggested they must bring their's into France, (when the edit of 1706 was petitioned for,) to make them Christians. He said, the distinction was difficult as to slavery, in its full extent, could not be tolerated here. Much consideration was necessary, to define how far the point should be carried. The Court must consider the great destriment to proprietors, there being so great a number in the ports of this kingdom, that many thousands of pounds would be lost to the owners, by setting them free. (A gentleman observed, no great danger ; for in a whole fleet, usually, there would not be six slaves.) As to France, the case stated decides no \textbf{[9]} farther than that kingdom ; and there freedom was claimed, because the slave had not been registered in the port where he entered, conformably to the edict of 1706. Might not a slave as well be freed by going out of Virginia to the adjacent country, where there are no slaves, if change to a place of contrary custom was sufficient? A statute by the Legislature, to subject the West India property to payments of debt, I hope, will be thought some proof ; another Act divests the African Company of their slaves, and vests them in the West India Company : I say, I hope, these are proofs the law has interfered for the maintenance of the trade in slaves, and the transferring of slavery. As for want of application properly to a Court of Justice ; a common servant may be corrected here by his master's private authority. Habeas corpus acknowledges a right to seize persons by force employed to serve abroad. A right of compulsion there must be, or the master will be under the ridiculous necessity of neglecting his proper business, by staying here to have their service, or must be quite deprived of those slaves he has been obliged to bring over. The case, as to service for life was not allowed, merely for want of a deed to pass it.

The Court approved Mr Alleyne's opinion of the distinction, how far municipal laws were to be regarded : instanced the right of marriage ; which properly solemnised, was in all places the same, but the regulations of power over children from it, and other circumstances, very various ; and advised, if the merchants thought it so necessary, to apply to Parliament, who could make laws.

Adjured till that day se'night.

\subsection{Mr Dunning}
Mr Dunning.---'Tis incumbent on me to justify Captain Knowles's detainer of the negro ; this will be effected, by proving a right in Mr Stewart ; even a supposed one : for till the matter was determined, it were somewhat unaccountable that a negro should depart his service, and put the means out of his power to trying that right to effect, by a flight out of the kingdom. I will explain what appears to me the foundation of Mr Stewart's claim. Before the writ of habeas corpus issued in the present case, there was, and there still is, a great number of slaves in Africa, (from whence the American plantations are supplied) who are saleable, and in fact sold. Under all these descriptions is James Somerset. Mr Stewart brought him over to England ; purposing to return to Jamaica, the negro chose to depart the service, as was stopped and detained by Captain Knowles, 'till his master should set sail and take him away to be sold in Jamaica. The gentlemen on the other side, to whom I impute no blame, but on the other hand much commendation, have advanced many ingenious propositions ; part of which are undeniably true, and part (as is usual in compositions of ingenuity) very disputable. 'Tis my misfortune \textbf{[10]} to address an audience, the greater part of which, I fear, are prejudiced the other way. But wishes, I am well convinced, will never enter into your Lordships minds, to influence the determination of the point : this cause must be what in fact and law it is : its fate, I trust, therefore, depends on fixed invariable rules, resulting by law from the nature of the case. For myself, I would not be understood to intimate a wish in favour of slavery, by any means ; nor on the other side, to be supposed maintainer of an opinion contrary to my own judgement. I am bound by duty to maintain those arguments which are most useful to Captain Knowles, as far as is consistent with the truth ; and if his conduct has been agreeable to the laws throughout, I am under a farther indispensable duty to support it. I ask no other attention than may naturally result from the importance of the question : less than this I have no reason to expect ; more, I neither demand nor with to have allowed. Many alarming apprehensions have been entertained of the consequence of the decision, either way. About 14~000 slaves, from the most exact intelligence I ma able to procure are at present here ; and some little time past, \marginpar{\textbf{505}} 166~914 in Jamaica ; there are, besides, a number of wild negroes in the woods. The computed value of a negro in those parts 50 \emph{l} a head. In the other islands I cannot state with the same accuracy, but on the whole they are about as many. The means of conveyance, I am told, are manifold ; every family almost brings over a greater number ; and will, be the decision on which side it may. Most negroes who have money (and that description I believe will include nearly all) make interest with the common sailors to be carried hitherto. There are negroes not falling under the proper denomination of any yet mentioned, descendants of the original slaves, the aborigines, if I may call them so ; these have gradually acquired a natural attachment to their country and situation ; in all insurrections they side with their masters : otherwise, the vast disproportion of the negroes to the whites, (not less probably than that of 100 to one) would have been fatal in its consequences. There are very strong and particular grounds of apprehension, if the relation in which they stand to their masters is utterly to be dissolved on the instant of their coming into England. Slavery, say the gentlemen, is an odious thing ; their name is : and the the reality ; it is were as one has defined, and the rest supposed it. If it were necessary to the idea and the existence of James Somerset, that is master, even here, might kill, nay, might eat him, might sell living or dead, might make him and his descendants property alienable, and thus transmissible to posterity ; this, how high soever my ideas may be of the duty of my profession, is what I should decline pretty much to defend or assert, for any purpose, seriously ; I should only speak of it to testify my contempt and abhorrence. But this is what as present I am not at all concerned in ; unless Captain Knowles, or Mr Stewart, have killed or eat him. Freedom has here been asserted as a natural right, and therefore unalienable and unrestrainable ; there is perhaps no branch of this right, but in some \textbf{[11]} at all times, and in all places at different times, has been restrained : nor could society otherwise be conceived to exist. For the great benefit of the public and individuals, natural liberty, which consists in doing what one likes, is altered to the doing of what one ought. The gentlemen who have spoken with so much zeal, have supposed different ways by which slavery commences ; but have omitted one, and rightly ; for it would have given a more favourable idea of the nature of that power against which they combat. We are apt (and great authorities support this way of speaking) to call those nations universally, whose integral policy we are ignorant of, barbarians ; (thus the Greeks, particularly, styled many nations, whose customs, generally considered, where far more justifiable and commendable than their own :) unfortunately, from calling them barbarians, we are apt to think them so, and draw conclusions accordingly. There are slaves in Africa by captivity in war, but the number far from great ; the country is divided into many small ,some great territories, who do, in their wars with one another, use this custom. There are of these people, men who have a sense of the right and value of freedom ; but who imagine that offences against society are punishable justly by the severe law of servitude. For crimes against property, a considerable addition is made to the number of slaves. They have a process by which the quantity of the debt is ascertained ; and if all the property of the debtor in goods and chattels is insufficient, he who has thus dissipated all he has besides, is deemed property himself ; the proper officer (sheriff we may call him) seizes the insolvent, and dispose of him as a slave. We don't contend under which of these the unfortunate man in question is ; but his condition was that of servitude in Africa ; the law of the land of that country disposed of him as property, with all the consequences of transmission and alienation ; the statutes of the British Legislature confirm this condition ; and thus he was a slave both in law and fact. I do not aim at proving these points ; not because they want evidence, but because they have not been controverted, to my recollection, and are, I think, incapable of denial. Mr Stewart, with this right, crossed the Atlantic, and was not to have the satisfaction of discovering, till after his arrival in this country, that all relation between him and the negro, as master and servant, was to be matter of controversy, and of long legal disquisition. A few worlds may be proper, concerning the Russian slave, and the proceedings of the House of Commons on that case. 'Tis not absurd in the idea, as quoted, nor improbably as matter of fact ; the expression as a kind of absurdity. I think, without any prejudice to Mr Stewart, or the merits of this cause, I may admit the utmost possible to be desired, as far as the case of that slave goes. The master and slave were bot, (or should have been at least) on their coming here, new \marginpar{\textbf{506}} creatures. Russian slavery, and even the subordination amongst themselves, in the degree they use it, is not here to be tolerated. Mr Alleyne justly observes, the municipal \textbf{[12]} regulations of one country are not binding on another ; but does the relation cease where the modes of creating it, the degrees in which it subsists, vary? I have not heard, I fancy, is there any intention to affirm, the relation of master and servant ceases here? I understand the municipal relations differ in different colonies, according to humanity, and otherwise. A distinction was endeavoured to the established between natural and municipal relations ; but the natural relations are not those which only attend the person of the man, political do so too ; with which the municipal are most closely connected : municipal laws, strictly, are those confined to a particular place ; political, are those in which the municipal laws of many States may and do concur. The relation of husband and wife, I think myself warranted in questioning, as a natural relation : does it subsist for life ; or to answer the natural purposes which may reasonably be supposed often to terminate sooner? Yet this is one of those relations which follow a many everywhere. If only natural relations had that property, the effect would be very limited indeed. In fact, the municipal laws are principally employed in determining the manner by which relations are created ; and which manner varies in various countries, and in the same country at different periods ; the political relation itself continuing usually unchanged by the change of place. There is but one form at present with us, by which the relation of husband and wife can be constituted ; there was a time when otherwise : I need not say other nations have their own modes, for that and other ends of society. Contract is not the only means, on the other hand, of producing the relation of master and servant ; the magistrates are empowered to oblige persons under certain circumstances to serve. Let me take notice, neither the air of England is too pure for a slave to breathe in, for the laws of England have rejected servitude. Villenage in this country is said to be worn out ; the propriety of the expression strikes me a little. Are the laws not existing by which it was created? A matter of more curiosity than use, it is, to enquire when that set of people ceased. The Statute of Tenures did not however abolish villenage in gross ; it left persons of that condition in the same state as before;  if their descendants are all dead, the gentlemen are right to say the subject of those laws is gone, but not the law ; if the subject revives, the law will lead the subject. If the Statute of Charles the 2d ever be repealed, the law of villenage revives in its full force. If my learned brother, the serjeant, or the other gentlemen who argued on the supposed subject of freedom, will go thro' an operation my reading assures me will be sufficient for that purpose, I shall claim them as property. I won't I assure them, make a rigorous use of my power ; I will neither sell them, eat them, nor part with them. It would be a great surprise, and some inconvenience, if a foreigner bringing over a servant, as soon as he got hither, must take care of his carriage, his horse, and himself, in whatever method he might have the luck to \textbf{[13]} invent. He must find his way to London on foot. He tells his servant, Do this ; the servant replies, Before I do it, I think fit to inform you, sir, the first step on this happy land sets all men on a perfect level ; you are just as much obliged to obey my commands. Thus neither superior, or inferior, both go without their dinner. We should find singular comfort, on entering the limits of a foreign country, to be thus at once divested of all attendance and all accommodation. The gentlemen have collected more reading than I have leisure to collect, or industry (I must own) if I had leisure : a very laudable pains have been taken, and very ingenious, in collecting the sentiments of other countries, which I shall into much regard, as affecting the point or jurisdiction of this Court. In Holland, so far from perfect freedom, (I speak from knowledge) there are, who without being conscious of contract, have for offences perpetual labour imposed, and death the condition annexed to non-performance. Either all the different ranks must be allowed natural, which is not readily conceived, or there are political ones, which cease not on change of soil. But in what manner is the negro to be treated? How far lawful to detain him? My footman, according to my agreement, is obliged to attend me from this city ; or he is not ; if not condition, that he shall not be obliged to attend, from hence he is obliged, and no injury done.

A servant of a sheriff, by the command of his master, laid hand gently on another servant of his master, and brought him before his master, who himself compelled the servant to his duty ; an action of assault and battery, and false imprisonment, was brought ; and the principal question was, on demurrer, whether the master could \marginpar{\textbf{507}} command the servant, tho' he might have justified his taking of the servant by his own hands? The convenience of the public is far better provided for, by this private authority of the master than if the lawfulness of the command were liable to be litigated every time a servant thought fit to be negligent or troublesome.

Is there a doubt, but a negro might interpose in the defence of a master, or a master in defence of a negro? If to all purposes of advantage, mutuality requires the rule to extend to those of disadvantage. 'Tis said, as not formed by contract, not restrain can be placed by contract. Which ever way it was formed, the consequences, good or ill, follow from the relation, not the manner of producing it. I may observe, there is an establishment, by which magistrates compel idle or dissolute persons, of various ranks and denominations, to serve. In the case of apprentices bound out by the parish, neither the trade is left to the choice of those who are to serve, nor the consent of the parties necessary ; no contract therefore is made in the former instance, none in the latter ; the duty remains the same. The case of contract for life quoted from the Yearbooks, was recognised as valid ; the solemnity only of an instru-\textbf{[14]}-ment judged requisite. Your Lordships, (this variety of service, with diverse other sorts, existing by law here,) have the opinion of classing him amongst those servants which e most resembles in condition : therefore, (it seems to me) are by law authorised to enforce a service for life in the slave, that being a part of his situation before his coming hither ; which as not incompatible but agreeing with our laws, may justly subsist here : I think, I might say, must necessarily subsist, as a consequence of a previous right in Mr Stewart, which our institution not dissolving, confirm. I don't insist on all the consequences of villenage ; enough is established from our cause, by supporting the continuance of the service. Much has been endeavoured, to raise a distinct, as the lawfulness of the negro's commencing slave, from the difficulty or impossibility of discovery by what means, under what authority, he became such. This, I apprehend, if a curious search were made, not utterly inexplicable ; nor the legality of his original servitude difficult to be proved. But to what end? Our Legislature, where it find a relation existing, supports it in all suitable consequences, without using to enquire how it commenced. A man enlists for not specified time ; the contract in construction of law, is for a year : the Legislature, when once the man is enlisted, interposes annually to continue him in the service, as long as the public has need of him. In times of public danger he is forced into the service ; the laws from thence forward find him a soldier, make him liable to all the burden, confer al lthe rights (if any rights there are of that state) and enforce all penalties of neglect of any duty in that profession, as much and as absolutely, as if by contract he had so disposed of himself. If the Court see a necessity in entering into the large field of argument, as to the right of the unfortunate man, and service appears to them deducible from a discussion of that nature to him, I neither doubt they will, nor wish they should not. As to the purpose of Mr Stewart and Captain Knowles, my argument does not require trover should lie, as for recovering of property, nor trespass : a form of action there is, the writ per quod servitium amisit, for loss of service, which the Court would have recognised ; if they allowed the means of suing a right, hey allowed the right. The opinion cited, to prove the negroes free on coming hither, only declares them not saleable ; does not take away their service. I would say, before I conclude, not for the sake of the Court, of the audience ; the matter now in question, interests the zeal for freedom of no person, if truly considered ; it being only, whether I must apply to a Court of Justice, (in a case, where if the servant was an Englishman I might use my private authority to enforce the performance of the service, according to its nature,) or may, without force or outrage, take my servant myself, or by another. I hope, therefore, I shall not suffer in the opinion of those honest passions are fired at the name of slavery. I hope I have not transgressed my duty of humanity ; nor doubt I your Lordships discharge of yours to justice.

\section{Plaintiff II}
\subsection{Serjeant Davy}
\textbf{[15]} Serjeant Davy.---My learned friend has thought proper to consider the question in the beginning of his speech, as of great importance : 'tis indeed so ; but not for those reasons principally assigned by him. I apprehend, my Lord, the honour of England, the honour of the laws of every Englishman ,here or abroad, is about concerned. he observes, the number of 14~000 or 15~000 ; if so, high time to put an end to the practice ; more especially, since they must be sent back as slaves, tho' servants here. The increase of such inhabitants, not interested in the prosperity of \marginpar{\textbf{508}} a country, is very pernicious ; in an island, which can, as such, not extend its limits, not consequently maintain more than a certain number of inhabitants, dangerous in excess. Money from foreign trade (or any other means) is not the wealth of a nation ; nor conduces any thing to support it, any farther than the produce of the earth will answer the demand of necessaries. In that case money enriches the inhabitants, as being the common representative of those necessaries ; but this presentation is merely imaginary and useless, if the increase of people exceeds the annual stock of provisions requisite for their subsistence. Thus, foreign superfluous inhabitants augmenting perpetually, are ill to be allowed ; a nation of enemies in the heart of a state, still worse. Mr Dunning availed himself of a wrong interpretation of the word natural : it was not used in the sense in which he thought fit to understand that expression ; 'twas used as moral, which no laws can supersede. All contracts, I do not venture to assert are of a moral nature ; but I know not any law to confirm an immoral contract and execute it. The contract of marriage is a moral contract, established for moral purposes, enforcing moral obligations ; the right of taking property by descent, the legitimacy of children ; (who in France are considered legitimate, tho' born before the marriage, in England not:) these, and many other consequences, flow from the marriage properly solemnised ; are governed by the municipal laws of that particular State, under whose institutions the contract and disposing parties live as subjects ; and by whose established forms they submit the relation to be regulated, so far as its consequences, not concerning the moral obligation, are interested. In the case of \emph{Thorn and Watkins}, in which your Lordship was counsel, determined before Lord Hardwicke, a man died in England, with effects in Scotland ; having a brother of the whole, and a sister of the half blood : the latter, by the laws of Scotland could not take. The brother applies for administration to take the whole estate, real and personal, into his own hands, for his own use ; the sister files a bill in Chancery. The then-Attorney General puts in answer for the defendant ; and affirms, the estate, as being in Scotland, as descending from a Scotsman, should be governed by that law. Lord Hardwicke overruled the objection against the sister's taking ; declared there was no pretence for it ; and spoke thus, to this effect, and nearly in the following \textbf{[16]} words---Suppose a foreigner has effects in our stock, and dies abroad ; they must be distributed according to the laws, not of the place where his effects where, but of that to which as a subject he belonged at the time of his death. All relations governed by municipal laws, must be so far dependent on them, that f the parties change their country the municipal laws give way, if contradictory to the political regulations of that other country. In the case of master and slave, being no moral obligation, but founded on principles and supported by practice, utterly foreign to the laws and customs of this country, the law cannot recognise such relation. The arguments founded on municipal regulations, considered in their proper nature, have been treated so fully, so learnedly, and ably, as scarce to leave any room for observations on that subject : anything I could offer to enforce, would rather appear to weaken the proposition, compared with the strength and propriety with which that subject has already been explained and urged. I am not concerned to dispute, the negro may contract to serve ; nor deny the relation between them, while he continues under his original proprietor's roof and protection. 'Tis remarkable, in all yer, for I have caused a search to be made as far as the 4th of Henry 8th, there is not one instance of a man's being held a villain who denied himself to be done ; nor can I find a confession of villenage in those times. [Lord Mansfield, the last confession of villenage extant, is in the 18th of Henry the 6th.] If the Court would acknowledge the relation of master and servant, it certainly would not allow the most exceptional part of slavery ; that of being obliged to remove, at the will of the master, from the protection of this land of liberty, to a country where there are now laws ; or hard laws to insult him. It will not permit slavery suspended for a while, suspended during the pleasure f the master. The instance of master and servant commencing without contract ; and that of apprentices against the will of the parties, (the letter found in its consequences exceedingly pernicious;) both these are provided by special statutes of our own municipal law. If made in France, or anywhere but here, they would not have been binding here. To punish not even a criminal for offences against the laws of another country ; to set free a galley slave, who is slave by his crime ; and make a slave of a negro, who is one, by his complexion ; is a cruelty and absurdity that I trust will never take place here : such as \marginpar{\textbf{509}} if promulged [promulgated], would make England a disgrace to all the nations under earth : for the reducing a man, guiltless of any offence against the laws, to the condition of slavery, the worst and most abject state, Mr Dunning has mentioned, what he is pleased to term philosophical and moral grounds, I think, or something to that effect, of slavery ; and would not by any means have us think disrespectfully of those nations, whom we mistakenly call barbarians, merely for carrying on that trade : for my part, we may be warranted, I believe, in affirming the morality or propriety of the practice does not enter their heads ; \textbf{[17]} they make slaves of whom they think fit. For the air of England ; I think, however, it has been gradually purifying ever since the reign of Elizabeth. Mr Dunning seems to have discovered so much, as he finds it changes to a slave into a servant ; tho' unhappily, he does not think it of efficacy enough to prevent that pestilent disease reviving, the instant the poor man is obliged to quit (voluntarily quits, and legally, it seems we ought to say,) this happy country. However, it has been asserted, and is now repeated by me, this air is too pure for a slave to breathe in : I trust, I shall not quit this Court without certain conviction of the truth of that assertion.

\section{Lord Mansfield}
\subsection{Preface}
Lord Mansfield.---The question is, if the owner had a right to detain the slave, for the sending of him over to be sold in Jamaica. In five or six cases of this nature, I have known it to be accommodated by agreement between the parties : on its first coming before me, I strongly recommended it here. But if the parties will have it decided, we must give our opinion. Compassion will not, on the other hand, nor inconvenience on the other, be to decide ; but the law : in which the difficulty will be principally from the inconvenience on both sides. Contract for sale of a slave is good here ; the sale is a matter to which the law properly and readily attaches, and will maintain the price according to the agreement. But here the person of the slave himself is immediately the object of enquiry ; which makes a very material difference. The now question is, whether any dominion, authority, or coercion can be exercise in this country, on a slave according to the American laws? 

The difficulty of adopting the relation, without adopting it in all its consequences, is indeed extreme ; and yet, many of those consequences are absolutely contrary to the municipal law of England. We have no authority to regulate the conditions in which law shall operate. On the other hand, should we think the coercive power cannot be exercised : 'tis now about fifty years since the opinion given by two of the greatest men of their own or any times, (since which no contract has been brought to trial, between the master and slaves ;) the service performed by the slaves without wages, is a clear indication they did not think themselves free by coming hither. The setting 14~000 or 15~000 men at once free loose by a solemn opinion, is much disagreeable in the effects it threatens. There is a case in Hobart, (\emph{Coventry and Woodfall},) where a man had contract to go as a mariner : but the now case will not come within that decision. Mr Stewart advances no claim on contract ; he rests his whole demand on a right to the negro as slave, and mentions the purpose of detainure to be the sending of him over to be sold in Jamaica. 

If the parties will have judgement, fiat justitia, ruat coelum, let justice be done whatever the consequence. 50 \emph{l} a head may not be a high price ; then a loss follows to the proprietors of over 700~000 \emph{l} sterling. How would the law stand with respect to their settlement ; their wages? \textbf{[18]} How many actions for any slight coercion by the master? We cannot in any of these points direct the law ; the law must rule us. In these particulars, it may be matter of weighty consideration, what provisions are made or set by law. Mr Stewart may end the question, by discharging or giving freedom to the negro. I did think at first to put the matter to a more solemn way of argument : but if my brothers agree, there seems no occasion. I do not imagine, after the point has been discussed on both sides so extremely well, any new light could be thrown on the subject. 

If the parties choose to refer it to the Common Pleas, they can give them that satisfaction whenever they think fit. An application to Parliament, if the merchants think the question of great commercial concern, is the best, and perhaps the only method of settling the point for the future. The Court is greatly obliged to the gentlemen of the Bar who have spoke on the subject ; and by whose care and abilities so much has been effected, that the rule of decision will be reduced to a very easy compass. I cannot omit to express particular happiness in seeing young men, just called to the Bar, have been able so much to profit by their reading. I think it right the matter should stand over ; and if we are called on for a decision, proper notice shall be given.

\subsection{Opinion}
Lord Mansfield.---On the part of Somerset, the case which we gave notice should be decided this day, the Court now proceeds to give its opinion. I shall recite the return to the writ of habeas corpus, as the ground of our determination ; omitting only words of form. 

The captain of the ship on board of which the negro was taken, makes his return to the writ in terms signifying that there have been, and still are, slaves to a great number in Africa ; and that the trade in them is authorised by the laws and opinions of Virginia and Jamaica ; that they are goods and chattels ; and as such, saleable and sold. That James Somerset, is a negro of Africa, and long before the return of the King's writ was brought to be sold, and was sold to Charles Stewart, Esq. then in Jamaica, and has not been manumitted since ; that Mr Stewart, having occasion to transact business, came over hither, with an intention to return ; and brought Somerset, to attend and abide with him, and to carry him back as soon as the business should be transacted. That such intention has been, and still continues ; and that the negro did remain till the time of his departure, in the service of his master Mr Stewart, and quitted it without his consent ; and thereupon, before the return of the King's writ, the said Charles Stewart did commit the slave on board to the ``Ann and Mary'', to save custody, to be kept till he should set sail, and then to be taken with him to Jamaica, and there sold as a slave. And this is the case why he, Captain Knowles, who was then and now is, commander of the above vessel, then and now lying in the river of \textbf{[19]} Thames, did the said negro, committed to his custody, detain ; and on which he now renders him to the orders of the Court. 

We pay all due attention to the opinion of Sir Philip Yorke, and Lord Chief Justice Talbot, whereby they pledged themselves to the British planters, for all the legal consequences of slaves coming over to this kingdom or being baptised, recognised by Lord Hardwicke, sitting as Chancellor on the 19th of October 1749, that trover would lie : that a notion had prevailed, if a negro came over, or became a Christian, he was emancipated, but not ground in law ; that he and Lord Talbot, when Attorney and Solicitor General, were of opinion, that no such claim for freedom was valid ; that tho' the Statute of Tenures had abolished villains regardant to a manor, yet he did not conceive but that a man might still become a villain in gross, by confessing himself such in open Court. We are so well agreed, that we think there is no occasion of having it argued (as I intimated an intention at first,) before all the Judges, as is usual, for obvious reasons, on a return to a habeas corpus ; the only question before us is, whether the cause on the return is sufficient. If it is, the negro must be remanded ; if it is not, he must be discharged. 

According, the return states, that the slave departed and refused to serve ; whereupon he was kept, to be sold abroad. So high an act of dominion must be recognised by the law of the country where it is used. The power of a master over his slave has been extremely different, in different countries. 

The state of slavery is such a nature, that it is in incapable of being introduced on any reasons, moral or political ; but only positive law, which preserves its force long after the reasons, occasion, and time itself from whence it was created, is erased from memory : it is so odious, that nothing can be suffered to support it, but positive law. Whatever inconveniences, therefore, may follow from a decision, I cannot say this case is allowed or approved by the law of England ; and therefore the black must be discharged.

\end{document}
